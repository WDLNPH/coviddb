\section{Relational Database Schema}
Convert your E/R diagram into a relational database schema. Make refinements to
the DB schema if necessary. Identify various integrity constraints such as primary
keys, foreign keys, functional dependencies, and referential constraints. Make sure
that your database schema is at least in 3NF.

\subsection{Relational Model from Warm-up Project}
This is the Relational schema we came up with for the warm-up project. Underline represents a primary key and \# represents a foreign key.
\begin{itemize}
    \item Person (\underline{person\_id}, medicare, first\_name, last\_name, address, city, postal\_code, province, citizenship, email, phone, dob)
    \item Parental (\underline{parental\_id}, \#person\_id, parent\_1\_id, parent\_2\_id, guardian\_id, child\_id)
    \item PublicHealthWorker (\underline{health\_worker\_id}, position, schedule,  \#health\_center\_id, \#person\_id)
    \item PublicHealthCenter (\underline{health\_center\_id}, phone, name, address, city, province, postal\_code, type, website)
    \item Diagnostic (\underline{diagnostic\_id}, diagnostic\_date, result, \#health\_worker\_id, \#health\_center\_id, \#person\_id)
    \item Patient (\underline{patient\_id}, \#person\_id)
    \item GroupZone (\underline{group\_id}, activity, name)
    \item GroupZonePersonPivot (\underline{group\_id}, \underline{person\_id})
\end{itemize}

\subsection{Relational Model with Updated Information}
For this project, we read the instructions carefully and started building on top of our schema to have a new one that includes all the new required information for our system.
\begin{itemize}
    \item Person (\underline{person\_id}, medicare, first\_name, last\_name, address, city, postal\_code, province, citizenship, email, phone, dob, password, api\_token)
    \item Patient (\underline{patient\_id}, \#person\_id, symptoms, symptoms\_history)
    \item Carer (\underline{parental\_id}, \#person\_id, parent\_1\_id, parent\_2\_id, guardian\_id, child\_id)
    \item PublicHealthWorker (\underline{health\_worker\_id}, position, schedule,\#health\_center\_id, \#person\_id)
    \item PublicHealthCenter (\underline{health\_center\_id}, phone, name, address, city, province, postal\_code, type, website, drive\_thru)
    \item Diagnostic (\underline{diagnostic\_id}, diagnostic\_date, result,\#health\_worker\_id, \#health\_center\_id, \#patient\_id)
    \item FollowupForm (\underline{form\_id}, \#health\_worker\_id, \#patient\_id)
    \item GroupZone (\underline{group\_id}, activity, name)
    \item Region (\underline{region\_id}, region\_name, city, postal\_code, province)
    \item Alert (\underline{alert\_id}, alert\_name, alert\_level, alert\_region, alert\_status, alert\_date, alert\_time, current\_alert, past\_alert, message, \#person\_id, recommendation)
    \end{itemize}

\subsection{Model's Normal Form}
It is safe to say that our model is at least in \textbf{1NF} for all the relations since all attributes are atomic and cannot contain more than one value. The only exception is symptoms \& symptoms\_history, which will be normalized later in this document.
\subsubsection{Person}
For the Person Relation we have the following non-trivial dependencies:\\
\begin{minipage}{\textwidth}
\begin{itemize}
    \item \{person\_id\} $\rightarrow$ first\_name
    \item \{person\_id\} $\rightarrow$ last\_name
    \item \{person\_id\} $\rightarrow$ medicare
    \item \{person\_id\} $\rightarrow$ password
    \item \{person\_id\} $\rightarrow$ address
    \item \{person\_id\} $\rightarrow$ postal\_code
    \item \{person\_id\} $\rightarrow$ citizenship
    \item \{person\_id\} $\rightarrow$ email
    \item \{person\_id\} $\rightarrow$ phone
    \item \{person\_id\} $\rightarrow$ dob
    \item \{person\_id\} $\rightarrow$ api\_token
    \item \{address\} $\not \rightarrow$ first\_name
    \item \{email\} $\not \rightarrow$ person\_id
\end{itemize}
\end{minipage}
person\_id cannot be determined, is not on the RHS of any of the FDs, making it a super-key and the only candidate key.

\begin{tcolorbox}
\textbf{So our $F$ is}:
\begin{multline}
$ F = \{ person\_id \rightarrow first\_name,last\_name, medicare, password,\\
address, postal\_code, citizenship, email, phone, dob, api\_token \} $
\end{multline}
\begin{multline}
$person\_id^+ = \{person\_id, first\_name,last\_name, medicare, password,\\
address,postal\_code, citizenship, email, phone, dob, api\_token \} $
\end{multline}

Key = $person\_id$\\
This relation is \textbf{in 2NF}.\\
This relation is \textbf{in 3NF}.\\
This relation is \textbf{in BCNF}.
\end{tcolorbox}
\newpage

\subsubsection{Carer}

For the Carer Relation we have the following non-trivial dependencies:\\

\begin{minipage}{\textwidth}
\begin{itemize}
    \item \{person\_id\} $\rightarrow$ child\_id
    \item \{person\_id\} $\rightarrow$ parent\_1\_id
    \item \{person\_id\} $\rightarrow$ parent\_2\_id
    \item \{person\_id\} $\rightarrow$ guardian\_id
    \item \{parent\_1\_id\} $\rightarrow$ child\_id
    \item \{parent\_2\_id\} $\rightarrow$ child\_id
    \item \{parent\_1\_id, parent\_2\_id\} $\rightarrow$ child\_id
    \item \{parent\_1\_id, parent\_2\_id, guardian\_id\} $\rightarrow$ child\_id
\end{itemize}
\end{minipage}

\begin{tcolorbox}
    \textbf{So our $F$ is}:
\begin{multline}
$F = \{ \{ person\_id \rightarrow child\_id, parent\_1\_id, parent\_2\_id, guardian\_id \}, \\
\{parent\_1\_id \rightarrow child\_id\} , \{parent\_2\_id \rightarrow child\_id\} , \\
\{parent\_1\_id, parent\_2\_id, guardian\_id \rightarrow child\_id \} \} $
\end{multline}
This relation is \textbf{not in 2NF} because we have partial dependencies, \\
For example $person\_id \rightarrow parent\_1\_id$ and we can split it into more tables. 
\end{tcolorbox}

\subsubsection{PublicHealthWorker}
For the PublicHealthWorker Relation we have the following non-trivial dependencies:\\
\\
\begin{minipage}{\textwidth}
\begin{itemize}
    \item \{health\_worker\_id\} $\rightarrow$ position
    \item\{health\_worker\_id\}  $\rightarrow$ schedule
    \item \{health\_worker\_id\}  $\rightarrow$ health\_center\_id
    \item \{health\_worker\_id\}  $\rightarrow$ person\_id
    \item \{health\_center\_id\}  $\not \rightarrow$ health\_worker\_id\\
\end{itemize}
\end{minipage}
\\
health\_worker\_id cannot be determined, is not on the RHS of any of the FDs, making it a super-key and the only candidate key.
\begin{tcolorbox}
    \textbf{So our $F$ is}:
$F = \{health\_worker\_id \rightarrow position, schedule, health\_center\_id, person\_id\}$\\
\\
$health\_worker\_id^+ = \{health\_worker\_id, person\_id, position, schedule, health\_center\_id \}$\\
\\
Key = $health\_worker\_id$\\
This relation is \textbf{in 2NF}.\\
This relation is \textbf{in 3NF}.\\
This relation is \textbf{in BCNF}.
\end{tcolorbox}

\subsubsection{PublicHealthCenter}
For the PublicHealthCenter Relation we have the following non-trivial dependencies:\\
\\
\begin{minipage}{\textwidth}
\begin{itemize}
    \item \{health\_center\_id\} $\rightarrow$ phone
    \item\{health\_center\_id\}  $\rightarrow$ name
    \item \{health\_center\_id\}  $\rightarrow$ address
    \item \{health\_center\_id\}  $\rightarrow$ city
    \item \{health\_center\_id\}  $\rightarrow$ province
    \item \{health\_center\_id\}  $\rightarrow$ postal\_code
    \item \{health\_center\_id\}  $\rightarrow$ type
    \item \{health\_center\_id\}  $\rightarrow$ website
    \item \{health\_center\_id\}  $\rightarrow$ drive\_thru
    \item \{postal\_code\}  $\rightarrow$ city
    \item \{postal\_code\}  $\rightarrow$ province
    \item \{city\}  $\rightarrow$ province
\end{itemize}
\end{minipage}

\begin{tcolorbox}
    \textbf{So our $F$ is}:
\begin{multline}
$F = \{ \{health\_center\_id \rightarrow phone, name, address, city, province, postal\_code, type, website,\\ 
drive\_thru\} , \{postal\_code \rightarrow city, province\}, \{city \rightarrow province\} \}$
\end{multline}
Key = $health\_worker\_id$ \\
This relation is \textbf{in 2NF}.\\
$postal\_code \rightarrow city, province$ \textbf{violates 3NF rules.}\\
It is non-trivial, LHS is not a superkey and RHS contains a non-key attribute.\\
This relation is \textbf{not in 3NF}.\\
This relation is \textbf{not in BCNF}.
\end{tcolorbox}
\newpage
\subsubsection{Diagnostic}
For the Diagnostic Relation we have the following non-trivial dependencies:\\
\\
\begin{minipage}{\textwidth}
\begin{itemize}
    \item  \{diagnostic\_id\}   $\rightarrow$ diagnostic\_date
    \item\ \{diagnostic\_id\}   $\rightarrow$ result
    \item  \{diagnostic\_id\}   $\rightarrow$ health\_worker\_id
    \item  \{diagnostic\_id\}   $\rightarrow$ health\_center\_id\
    \item  \{diagnostic\_id\}   $\rightarrow$ patient\_id
    \item  \{patient\_id\,health\_worker\_id\} $\not \rightarrow$ diagnostic\_id
    \item  \{patient\_id\,health\_worker\_id\} $\not \rightarrow$ result
\end{itemize}
\end{minipage}

\begin{tcolorbox}
\textbf{So our $F$ is}:
$F = \{diagnostic\_id \rightarrow diagnostic\_date, result, health\_worker\_id, health\_center\_id\}$\\
Key = $diagnostic\_id$\\
diagnostic\_id cannot be determined, is not on the RHS of any of the FDs, making it a super-key and the only candidate key.\\
This relation is \textbf{in 2NF}.\\
This relation is \textbf{in 3NF}.\\
This relation is \textbf{in BCNF}.
\end{tcolorbox}

\subsubsection{FollowUpForm}
For the FollowUpForm Relation we have the following non-trivial dependencies:\\
\\
\begin{minipage}{\textwidth}
\begin{itemize}
    \item  \{form\_id\}   $\rightarrow$ health\_worker\_id
    \item  \{form\_id\}   $\rightarrow$ patient\_id
\end{itemize}
\end{minipage}
form\_id cannot be determined, is not on the RHS of any of the FDs, making it a super-key and the only candidate key.
\begin{tcolorbox}
    \textbf{So our $F$ is}:
$F = \{form\_id \rightarrow health\_worker\_id, patient\_id\}$\\
\\
$form\_id^+ = \{form\_id, health\_worker\_id, patient\_id\}$\\
\\
Key = $form\_id$\\
This relation is \textbf{in 2NF}.\\
This relation is \textbf{in 3NF}.\\
This relation is \textbf{in BCNF}.
\end{tcolorbox}
\newpage
\subsubsection{GroupZone}
For the Groupzone Relation we have the following non-trivial dependencies:\\
\\
\begin{minipage}{\textwidth}
\begin{itemize}
    \item  \{group\_id\}   $\rightarrow$ activity
    \item  \{group\_id\}   $\rightarrow$ name
\end{itemize}
\end{minipage}
group\_id cannot be determined, is not on the RHS of any of the FDs, making it a super-key and the only candidate key.
\begin{tcolorbox}
    \textbf{So our $F$ is}:
$F = \{group\_id \rightarrow activity, name \}$\\
\\
$group\_id^+ = \{group\_id, activity, name\}$\\
\\
Key = $group\_id$\\
This relation is \textbf{in 2NF}.\\
This relation is \textbf{in 3NF}.\\
This relation is \textbf{in BCNF}.
\end{tcolorbox}
\subsubsection{Region}
For the Region Relation we have the following non-trivial dependencies:\\
\\
\begin{minipage}{\textwidth}
\begin{itemize}
    \item  \{region\_id\}   $\rightarrow$ region\_name
    \item  \{region\_id\}   $\rightarrow$ city
    \item  \{region\_id\}   $\rightarrow$ postal\_code
    \item  \{region\_id\}   $\rightarrow$ province
    \item \{postal\_code\}  $\rightarrow$ city
    \item \{postal\_code\}  $\rightarrow$ province
    \item \{city\}  $\rightarrow$ province

\end{itemize}
\end{minipage}
\begin{tcolorbox}
    \textbf{So our $F$ is}:
\begin{multline}
$F = \{ \{region\_id \rightarrow region\_name, city, postal\_code, province\}, \\
\{postal\_code \rightarrow city, province\}, \{city \rightarrow province\} \}$
\end{multline*}
\\
This relation is \textbf{in 2NF}.\\
$postal\_code \rightarrow city, province$ \textbf{violates 3NF rules.}\\
It is non-trivial, LHS is not a superkey and RHS contains a non-key attribute.\\
This relation is \textbf{not in 3NF}.\\
This relation is \textbf{not in BCNF}.
\end{tcolorbox}

\newpage
\subsubsection{Alert}
For the Alert Relation we have the following non-trivial dependencies:\\
\\
\begin{minipage}{\textwidth}
\begin{itemize}
    \item \{alert\_id\}   $\rightarrow$ alert\_name
    \item \{alert\_id\}   $\rightarrow$ alert\_level
    \item \{alert\_id\}   $\rightarrow$ alert\_region
    \item \{alert\_id\}   $\rightarrow$ alert\_status
    \item \{alert\_id\}   $\rightarrow$ alert\_date
    \item \{alert\_id\}   $\rightarrow$ alert\_time
    \item \{alert\_id\}   $\rightarrow$ current\_alert
    \item \{alert\_id\}   $\rightarrow$ past\_alert
    \item \{alert\_id\}   $\rightarrow$ message
    \item \{alert\_id\}   $\rightarrow$ person\_id
    \item \{alert\_id\}   $\rightarrow$ recommendation
    \item \{alert\_region\}   $\rightarrow$ alert\_status
    \item \{alert\_region\}   $\rightarrow$ alert\_level
    \item \{alert\_region\}   $\rightarrow$ message
    \item \{message\}   $\rightarrow$ recommendation
    \item \{person\_id} $\rightarrow$ message
\end{itemize}
\end{minipage}

\begin{tcolorbox}
    \textbf{So our $F$ is}:
\begin{multline}
$F = \{ \{alert\_id \rightarrow alert\_name, alert\_level, alert\_region, alert\_status, alert\_date, \\ alert\_time, current\_alert, past\_alert, message, person\_id, recommendation\} , \\
\{alert\_region \rightarrow alert\_status, alert\_level, message\}, \\
\{message \rightarrow recommendation\}, \{person\_id \rightarrow message \}\}$
\end{multline}
Key = $alert\_id$\\
This relation is \textbf{in 2NF}.\\
$alert\_region \rightarrow alert\_status, alert\_level, message$ \textbf{violates 3NF rules.}\\
It is non-trivial, LHS is not a superkey and RHS contains a non-key attribute.\\
This relation is \textbf{not in 3NF}.\\
This relation is \textbf{not in BCNF}.
\end{tcolorbox}
\newpage
\subsection{Few Relation instances}
\subsubsection{Person}
api\_token and password are randomly generated numbers or hashed strings. Shortened them for the report.
\begin{table}[H]
\centering
\begin{adjustbox}{width=1.2\textwidth,center=\textwidth}
\begin{tabular}{|l|l|l|l|l|l|l|l|l|l|l|l|l|l|} 
        \hline
        person\_id & medicare & first\_name & last\_name & address & city & postal\_code & province & citizenship & email & phone & dob & password & api\_token \\
        \hline 
4 & GBUJ46611976 & Anne & Paquette & 6577 Sébastien Plains & Sainte-Catherine & G3N 0O9 & QC & KN & bstpierre@lapierre.com & 1-882-823-4013, & 1977-02-08 & KxsFhruN & gwh0XS \\    
16 & EEKJ96363064 & Marc & Gauthier & 756 Tessier Prairie & Montreal & H1H 7D2 & QC & TF & brigitte92@lambert.com & 470-997-1115, & 1990-06-29 & zd9Znke & xO3TUV\\        
        \hline
\end{tabular}
\end{adjustbox}
\caption{Un-normalized person relation instance}
\end{table}
\begin{table}[H]
\centering
\begin{adjustbox}{width=1.2\textwidth,center=\textwidth}
\begin{tabular}{|l|l|l|l|l|l|l|l|l|l|l|l|l|l|} 
        \hline
        person\_id & medicare & first\_name & last\_name & address & postal\_code & postal\_code\_id & citizenship & email & phone & dob & password & api\_token \\
        \hline 
4 & GBUJ46611976 & Anne & Paquette & 6577 Sébastien Plains & G3N 0O9 & G3N & KN & bstpierre@lapierre.com & 1-882-823-4013, & 1977-02-08 & KxsFhruN & gwh0XS \\    
16 & EEKJ96363064 & Marc & Gauthier & 756 Tessier Prairie & H1H 7D2 & H1H & TF & brigitte92@lambert.com & 470-997-1115, & 1990-06-29 & zd9Znke & xO3TUV\\        
        \hline
\end{tabular}
\end{adjustbox}
\caption{Normalized person relation instance}
\end{table}


\subsubsection{Patient}
\begin{table}[H]
\centering
\begin{adjustbox}{width=0.5\textwidth,center=\textwidth}
\begin{tabular}{|l|l|l|l|} 
        \hline
        patient\_id & person\_id & symptoms & symptoms\_history \\ 
        \hline
        1 & 11 & cough,fever & cough,fever,loss of taste\\
        6 & 16 & NULL & NULL\\
        \hline
\end{tabular}
\end{adjustbox}
\caption{Un-normalized patient relation instance}
\end{table}

\begin{table}[H]
\centering
\begin{adjustbox}{width=0.2\textwidth,center=\textwidth}
\begin{tabular}{|l|l|} 
        \hline
        patient\_id & person\_id \\ 
        \hline
        1 & 11 \\
        6 & 16 \\
        \hline
\end{tabular}
\end{adjustbox}
\caption{Normalized patient relation instance}
\end{table}



\subsubsection{Carer}
\begin{table}[H]
\centering
\begin{adjustbox}{width=0.6\textwidth,center=\textwidth}
\begin{tabular}{|l|l|l|l|l|l|} 
        \hline
        parental\_id & person\_id & parent\_1\_id & parent\_2\_id & guardian\_id & child\_id\\ 
        \hline
        1 & 4 & 5 & 7 & NULL & 4\\
        2 & 9 & NULL & NULL & 11 & 9\\
        \hline
\end{tabular}
\end{adjustbox}
\caption{Un-normalized carer relation instance}
\end{table}

\begin{table}[H]
\centering
\begin{adjustbox}{width=0.2\textwidth,center=\textwidth}
\begin{tabular}{|l|l|} 
        \hline
        child\_id & parent\_id \\ 
        \hline
        4 & 5 \\
        4 & 7 \\
        9 & 11 \\
        \hline
\end{tabular}
\end{adjustbox}
\caption{Normalized carer relation instance}
\end{table}

\newpage
\subsubsection{PublicHealthWorker}
\begin{table}[ht]
\centering
\begin{adjustbox}{width=1\textwidth,center=\textwidth}
\begin{tabular}{|l|l|l|l|l|} 
        \hline
        health\_worker\_id & person\_id & position & schedule & health\_center\_id\\ 
        \hline
        1 & 18 & Doctor & Mon: 19:00-03:00, Tue: 07:00-15:00 & 4\\
        2 & 12 & Nurse & Thu: 04:00-12:00, Fri: 17:00-01:00, Sat: 09:00-17:00 & 5\\
        \hline
\end{tabular}
\end{adjustbox}
\caption{Un-normalized PublicHealthWorker relation instance}
\end{table}

\begin{table}[ht]
\centering
\begin{adjustbox}{width=1\textwidth,center=\textwidth}
\begin{tabular}{|l|l|l|l|l|} 
        \hline
        health\_worker\_id & person\_id & position & schedule & health\_center\_id\\ 
        \hline
        1 & 18 & 1 & Mon: 19:00-03:00, Tue: 07:00-15:00 & 4\\
        2 & 12 & 2 & Thu: 04:00-12:00, Fri: 17:00-01:00, Sat: 09:00-17:00 & 5\\
        \hline
\end{tabular}
\end{adjustbox}
\caption{Normalized PublicHealthWorker relation instance}
\end{table}

\subsubsection{Position}
\begin{table}[H]
\centering
\begin{adjustbox}{width=0.2\textwidth,center=\textwidth}
\begin{tabular}{|l|l|} 
        \hline
        position\_id & position \\ 
        \hline
        1 & Doctor \\
        2 & Nurse \\
        3 & Intern \\
        \hline
\end{tabular}
\end{adjustbox}
\caption{Normalized Position relation instance}
\end{table}


\subsubsection{PublicHealthCenter}
\begin{table}[H]
\centering
\begin{adjustbox}{width=1.2\textwidth,center=\textwidth}
\begin{tabular}{|l|l|l|l|l|l|l|l|l|l|} 
        \hline
        health\_center\_id & phone & name & address & city & province & postal\_code & type & website & drive\_thru\\ 
        \hline
        4 & (828) 624-4739 & Boucher-Jacques & 1717 Boucher Well & Saint-Laurent & QC & H4T 4V7 & Hospital & www.simard.biz & 0\\
        5 & +1.903.423.4085 & Guay Ltd & T448 Girard Lane Suite 134 & Laval & QC & H7N 5N3 & Clinic & www.gangon.info & 1\\
        \hline
\end{tabular}
\end{adjustbox}
\caption{Un-normalized PublicHealthCenter relation instance}
\end{table}

\begin{table}[H]
\centering
\begin{adjustbox}{width=1.2\textwidth,center=\textwidth}
\begin{tabular}{|l|l|l|l|l|l|l|l|l|l|} 
        \hline
        health\_center\_id & phone & name & address & postal\_code\_id& postal\_code & type & website & method & drive\_thru\\ 
        \hline
        4 & (828) 624-4739 & Boucher-Jacques & 1717 Boucher Well & H4T & H4T 4V7 & Hospital & www.simard.biz & appointment & 0\\
        5 & +1.903.423.4085 & Guay Ltd & T448 Girard Lane Suite 134 & H7N & H7N 5N3 & Clinic & www.gangon.info & walk-in & 1\\
        \hline
\end{tabular}
\end{adjustbox}
\caption{Normalized PublicHealthCenter relation instance}
\end{table}

\subsubsection{PostalCode}
\begin{table}[H]
\centering
\begin{adjustbox}{width=0.25\textwidth,center=\textwidth}
\begin{tabular}{|l|l|} 
        \hline
        postal\_code\_id & city\_id \\ 
        \hline
        V1R & 229 \\
        V2K & 244 \\
        V7A & 304 \\
        \hline
\end{tabular}
\end{adjustbox}
\caption{Normalized PostalCode relation instance}
\end{table}

\subsubsection{City}
\begin{table}[H]
\centering
\begin{adjustbox}{width=0.35\textwidth,center=\textwidth}
\begin{tabular}{|l|l|l|} 
        \hline
        city\_id & city & region\_id \\ 
        \hline
        229 & Trail & 95 \\
        244 & Prince George North & 101 \\
        304 & Richmond South  & 126\\
        \hline
\end{tabular}
\end{adjustbox}
\caption{Normalized City relation instance}
\end{table}

\subsubsection{Region}
\begin{table}[H]
\centering
\begin{adjustbox}{width=0.45\textwidth,center=\textwidth}
\begin{tabular}{|l|l|l|l|l|} 
        \hline
        region\_id & region\_name & city & postal\_code & province \\ 
        \hline
        95 & Trail & Trail & V1R & British Columbia \\
        101 & Prince George & Prince George North & V2K & British Columbia \\
        126 & Richmond & Richmond South & V7A & British Columbia\\
        \hline
\end{tabular}
\end{adjustbox}
\caption{Un-normalized Region relation instance}
\end{table}

\begin{table}[H]
\centering
\begin{adjustbox}{width=0.45\textwidth,center=\textwidth}
\begin{tabular}{|l|l|l|l|l|} 
        \hline
        region\_id & region\_name & alert\_id & province\_id \\ 
        \hline
        95 & Trail & 1 & BC \\
        101 & Prince George North & 1 & BC\\
        126 & Richmond South  & 1 & BC\\
        \hline
\end{tabular}
\end{adjustbox}
\caption{Normalized Region relation instance}
\end{table}


\subsubsection{Province}
\begin{table}[H]
\centering
\begin{adjustbox}{width=0.35\textwidth,center=\textwidth}
\begin{tabular}{|l|l|l|l|l|} 
        \hline
        province\_id & province \\ 
        \hline
        BC & British Columbia\\
        SK & Saskatchewan\\
        QC & Quebec\\
        \hline
\end{tabular}
\end{adjustbox}
\caption{Normalized Province relation instance}
\end{table}